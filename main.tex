\documentclass{article}
\usepackage{amsmath}
\usepackage{amssymb}
\usepackage{setspace}
\usepackage{geometry}
\geometry{a4paper, margin=2.5cm}
\linespread{1.3}


\begin{document}

\begin{center}
  \section*{PART1}
  \section*{定数と方程式の定義}
\end{center}
\setcounter{section}{1}
\setcounter{subsection}{-1}
\subsection{はじめに}
1976年米国標準大気は、地表面から1000kmまでの地球大気を、穏やかな太陽活動期に存在すると仮定した、理想化された定常状態の表現である。

地表面から51地勢ポテンシャルキロメートル (km') までの高さについては、この標準の表は、1962年米国標準大気(COESA 1962)のものと同一であり、伝統的な定義に基づいている。
これらの定義は、特に20 km' 未満の高さについては、その高さ領域内の観測から今日利用可能な膨大な量の大気データを必ずしも平均的に表しているとは限らない。

51 km' から 84.852 km' (すなわち、51.413 から 86 幾何キロメートル)までの高さについては、表は、伝統的なタイプの定義パラメータで表される、今日の最新の大気データの平均に基づいている。
これには、線形に分割された温度-高さプロファイル、および空気がいくつかの構成気体の均質な混合物として扱われる静水圧平衡の仮定が含まれる。

ただし、それより高い高度では、解離および拡散プロセスによって均質性からの著しいずれが生じるため、標準を支配する定義は、より低い高度で使用される定義よりも洗練されている。

この高高度領域では、混合大気に適用される静水圧方程式は、個々のガス種のフラックスの垂直成分に対するより一般的な方程式(Colegrove et al. 1965; Keneshea and Zimmerman, 1970)に道を譲り、これは高さに伴う組成の相対的な変化を考慮に入れている。
このフラックス方程式は、大気ガスが86 km未満の場合のように、十分に混合された状態を維持する場合の特殊なケースでは、静水圧方程式に簡略化される。

86 km から 1000 km の間の温度-高さプロファイルは、低い高度のように、一連の線形関数として表されるわけではない。

むしろ、観測に対する合理的な近似を提供するだけでなく、高さ領域全体にわたって高さに関して連続的な一次導関数をもたらすように選択された4つの連続する関数によって定義される。

さまざまな種類の観測データが、この温度-高さプロファイルのさまざまなセグメントを独立して決定するための基礎となる。
110 km から 120 km の間の高さで観測された温度は、特にその領域の温度-高さ関数の選択に制限を課す上で重要であり、150 km 以上の観測された密度は、86 km のすぐ上の低温等温層の温度と範囲の両方の選択に強く影響を与えた。

いくつかの独立したデータセットに基づいているにもかかわらず、この標準の表の数学的な再現性の目的のために、地表面から1000 kmまでの一連の連続する高さ関数で温度を表すことが望ましく、各連続する関数の式は、関連する高さ間隔に特有の特定の項と係数だけでなく、先行する関数の終点値に依存する。

この合計温度-高さプロファイルは、すべての補助的な必要な定数、係数、および関数とともに、基本的な連続性モデル(すなわち、静水圧方程式と運動方程式)に適用され、1976年の米国標準大気を定義する。
観測されたデータの観点から正当性を示すことなく、この定義の仕様を示すことが、セクション1の目的である。

\subsection{国際単位系}

1976年の米国標準大気は、国際単位系(SI)(Mechtley 1973)で定義されている。

適用可能な基本および誘導されたSI単位、ならびにこの標準で使用される非標準のメートル単位と英語単位の記号、名前、および関連する量のリストを表1に示す。

\subsection{基本的な仮定と公式}
\subsubsection{採用された定数}

計算の目的のために、地球の大気に適切なさまざまな定数の数値を確立する必要がある。
採用された定数は、3つのカテゴリにグループ化される。

カテゴリIには、物理科学および化学科学の多くの分野に共通する定数が含まれており、ここでは基本的な定数と見なされる。
これらの定数の中には、$M_i$ が i 番目のガス種の分子量を示す場合のように、多価であるものもある。
カテゴリIには、3つの単一値定数と1つの多価定数が含まれる。

カテゴリIIには、カテゴリIの定数と適切な方程式のセットに加えて、1976年標準大気の86 km未満の部分を定義するのに十分な定数が含まれる。

このカテゴリには、9つの単一値定数と3つの多価定数が含まれる。カテゴリIIIには、カテゴリIおよびカテゴリIIの定数と関連する方程式に加えて、そのセットの拡張とともに、1976年標準大気の86 kmを超える部分を定義するために必要な残りのすべての定数が含まれる。このカテゴリには、$7$ つの単一値定数と $11$ の多価定数が含まれる。

定数は、適切な次元と記号とともに、表2の3つの連続するセクションにカテゴリ別にリストされている。

各定数の値の定義と権威は、表形式のリストとは別に議論されている。多価定数は、1つの例外を除いて、一般的な記号と次元のみが表2にリストされており、これらの定数の複数の値、すなわち、いくつかのガス種ごとの値、またはいくつかの高さレベルごとの値が表4から7にリストされている。
\\\\
一次定数の採用値の説明:
\\\\
カテゴリI定数\\

$k$\\
ボルツマン定数 $k = 1.380622 \times 10^{-23} \, \mathrm{Nm/K}$ は、理論的には比 ${R^*}/{N_A}$ に等しく、Mechtley (1973) によって引用された、炭素-12スケールと一致する値を持っている。

$M_i$\\
表3にリストされている分子量 $M_i$ の値のセットは、1961年に採用された炭素-12同位体スケール $C^{12} = 12$ に基づいている。このスケールは、国際純正・応用化学連合のモントリオール会議で1961年に採用された。

$M_A$\\
アボガドロ定数 $N_A = 6.022169 \times 10^{26} \, \mathrm{kmol}^{-1}$ は、炭素-12スケールと一致しており、Mechtley (1973) によって引用された値である。

$R^{*}$\\
気体定数 $R^* = 8.31432 \times 10^{3} \, \mathrm{N \cdot m / (kmol \cdot K)}$ は、炭素-12スケールと一致しており、1962年の標準で使用された値である。この値は、$k$ と $N_A$ の引用された値と正確には一致していない。

カテゴリII定数\\
\\$F_i$\\
表3にリストされている体積パーセント濃度 $F_i$ の値のセットは、海面での乾燥空気を構成するいくつかのガス種の相対濃度を表すと想定されている。
これらの値は、1962年の標準(COESA 1962)で与えられた値と同一であり、Keeling(1960)による $CO_2$ 測定に基づいたわずかな修正を除いて、これらの値はGlueckauf(1951)によって与えられた値と同じであり、Paneth(1939)の以前の研究に基づいている。
\\$g_0$\\
量 $g_0 (= 9.80665 \, \mathrm{m/s^2})$ は、この標準に採用された重力加速度の海面値を表している。この値は、元々1901年に国際度量衡委員会によって45°緯度で採用されたものであり、約1万分の5高いことが示されているが(List 1968)、この値は、45°32'33" の緯度に正確に適用される場合でも、気象学および一部の標準大気で45°緯度に関連付けられた値として残っている。\\
\\$g_0\text{$'$}$\\
標準地勢ポテンシャルメートルを幾何高度に関連付けるために選択された次元定数 $g_0\text{$'$}$ は、数値的には $g_0$ に等しいが、次元は適切に異なる。
この定数は、暗黙のうちに、1標準地勢ポテンシャルメートルを、1キログラムの潜在エネルギーを9.80665ジュールだけ増加させるために、1キログラムを持ち上げなければならない垂直増分として定義する。
この垂直増分の幾何学的長さは、$g$ の高さに依存する値に反比例する。\\
\\$H_b$\\
表4にリストされている地勢ポテンシャル高さの値 $H_b$ のセットの各メンバーは、8つの連続する大気層の1つの基部を表している。
$H_b$ と $L_{Mb}$ の値のペアは、一部は伝統に基づいており、一部は今日の観測に基づいている。
これらのペアの最初の5つは、1962年の標準の最初の5つの層と同一であり、$H_b$ と $L_{M,b}$ の残りの2つの値は、現在入手可能な大気データへの合理的な適合を提供するように新たに選択されている。
関連するセットの最初の2つの値は、最も初期の航空標準大気の1つ(Toussaint 1919)に由来し、最初の米国標準大気(Diehl 1925)で近似された。\\
\\$L_{M,b}$\\
7つの勾配 $L_{M,b} = dT_M/dH$ [すなわち、地勢ポテンシャル $H$ に関する分子スケール温度 $T_M$(Minzner and Ripley 1956)の勾配] のセットの各メンバー(表4にリスト)は、関連する層全体で適切な固定値を表している。$H_b$ から $H_{b+1}$ まで。\\
\\$P_0$\\
標準海面大気圧 $P_0$ は、$1.013250 \times 10^{5} \, \mathrm{Pa}$ (または N/m$^2$) に等しく、1947年に国際気象機関の決議164で採用され、高さ 0.760 m の水銀柱によって及ぼされる圧力に対応し、その密度は $1.35951 \times 10^{4} \, \mathrm{kg/m^3}$ であり、重力加速度は $9.80665 \, \mathrm{m/s^2}$ である。
この同等性の定義は、1948年に国際度量衡委員会によって採用された。\\
\\$r_0$\\
任意の緯度で地勢ポテンシャルを計算する目的のための有効な地球半径は、Harrison(1968)によって与えられた方程式から容易に得られる。
この標準で使用される $r_0 (= 6356.766 \, \mathrm{km})$ の値は、$g = 9.80665 \, \mathrm{m/s^2}$ の緯度に対応する。\\
\\$T_0$\\
標準海面温度 $T_0$ は $288.15 \, \mathrm{K}$ である。
この値は、2つの国際協定に基づいている。
これらの最初の協定は、国際航空委員会(International Commission for Air Navigation)の決議192であり、1924年に国際標準大気の海面温度として $15^\circ \, \mathrm{C}$ を採用した。
この値は、その日以来、既知のすべての標準大気で変更されずに保持されている。
2番目の協定は、1954年の第10回度量衡に関する会議であり、ケルビン温度スケールの固定点を、標準海面圧での氷点温度より0.01 K高い、三重点温度 $273.16 \, \mathrm{K}$ に設定した。\\
\\$S$\\
サザーランド定数 $S = 110 \, \mathrm{K}$ (Hilsenrath et al. 1955) は、動粘度の経験式における定数である。\\\\
\\$\beta$\\
量 $\beta = 1.458 \times 10^{6} \, \mathrm{kg/(s \cdot m \cdot K^{1/2})}$ (Hilsenrath et al. 1955) は、動粘度の式における定数である。\\
\\$\gamma$\\
一定圧力での空気の比熱と一定体積での空気の比熱の比は、無次元量であり、採用値は $\gamma = 1.400$ である。
これは、1948年にトロントで開催された国際気象機関の航空委員会によって採用された値である。\\
\\$\sigma$\\
分子の平均有効衝突直径 $\sigma (= 3.65 \times 10^{-10} \, \mathrm{m})$ は、ガス種と温度によって変化する量である。
採用された値は、乾燥した海面大気に適用されると想定されている。
85 kmを超える高度では、大気組成の変化により、採用された値の妥当性は高度の上昇とともに低下する(Hirschfelder et al. 1965; Chapman and Cowling 1960)。このため、$\sigma$ を含む量の表における有効数字の数は、86 kmを超える高度で他の表形式の量に使用される数から削減されている。
\\カテゴリIII定数\\
\\$a_i$\\
量 $a_i$ は、表6にリストされている種に依存する係数の5つの値のセットを表している。これらの値のそれぞれは、関連するガス種の高さに依存する分子拡散係数 $D_i$ を指定するための特定の関数で使用される(下の $b_i$ を参照)。\\
\\$b_i$\\
量 $b_i$ は、表6にリストされている種に依存する指数の5つの値のセットを表している。これらの値のそれぞれは、対応する $a_i$ の値とともに、関連するガス種の高さに依存する分子拡散係数を指定するための式 (8) で使用される。この標準に採用された $a_i$ と $b_i$ の特定の値は、$D$ の高さ変動を生成するように選択されている。
\\$K_7$\\
量 $K_7 = 1.2 \times 10^7 \, \mathrm{m^2/s}$ は、$Z_1 = 86 \, \mathrm{km}$ で、86 から 197 km までの高さ間隔における渦拡散係数 $K$ の採用値である。91 kmから始まり、115 kmまで拡張して、$K$ の値は式 (7b) によって定義される。115 km での $K$ の値は $K_{10}$ に等しい。\\
\\$K_{10}$\\
量 $K_{10} = 0.0 \, \text{m}^2/\text{s}$ は、$Z_{10} = 120$ で、115 km から 1000 $\text{m}^2/\text{s}$ までの高さ間隔で最も高い渦拡散係数 $K$ の採用値である。\\
\\$L_{K,b}$\\
表5にリストされている2値勾配のセット $L_{K,b} = dT/dZ$ は、利用可能な観測値を表すように標準用に特別に選択された。これらの $L_{K,b}$ の2つの値のそれぞれは、$Z_b$ が基部である対応する層の全体の範囲に関連付けられており、上部は $Z_{b+1}$ である。\\\\
\\$n(O)_7$\\
量 $N(O)_7$, (= $8.6 \times 10^{13} \, \text{m}^{-3}$), は、$Z_7 = 86 \, \text{km}$ に存在するためにこの標準に割り当てられた原子状酸素の数密度である。この値は、86 km での原子状酸素、水素原子の他の数密度、および $N_2$、$O_2$、$Ar$、および $He$ の密度とともに使用される(付録Aを参照)。\\
\\$n(H)_{11}$\\
量 $N(H)_\infty$, (= $1.0 \times 10^{10} \, \text{m}^{-3}$) は、高さ $Z_\infty = 500 \, \text{km}$ での原子状水素の割り当てられた数密度であり、150 km から 1000 km の間の原子状水素の高さプロファイルを計算する際の基準値として使用される。\\
\\$q_i$\\
量 $q_i$ は、最初の6つの種に依存する係数または項のセット(すなわち、$q_i$、$Q_i$、$u_i$、$U_i$、$w_i$、および $W_i$ のセット)を表しており、これら6つのセットの対応するメンバーはすべて、特定のガス種の垂直フラックス方程式における垂直輸送項 $v_i/(D_i+K)$ の経験式 [式 (37)] で同時に使用される。これら6つのセットすべての種に依存する値は、関連するガス種の数密度プロファイルを、150 および 450 $\mathrm{km}$、ならびに原子状酸素の場合は 97 $\mathrm{km}$ で特定の境界条件に調整するために、この標準用に選択されている。これらの境界条件はすべて、観測されたまたは想定された平均状態を表している。これらの6つの値のセットは、表7にリストされている。\\
\\$Q_i$\\
量 $Q_i$ は、$q_i$ adove とともに説明されている6つの定数のセットの2番目のセットを表している。\\
\\$T_9$\\
量 $T_9 (=240.0\text{K})$ は、$Z_9 = 110\mathrm{km}$ での運動エネルギーを表す。この温度は、勾配 $L_{\text{K},9}$ (=12K/km) とともに採用され、この標準の110 km から 120 kmの間の $T(Z)$ の線形セグメントを生成する。$T(Z)$ のこのセグメントは、対応する高さ領域の観測された温度と高さのデータの平均を表している。\\
\\$T_\infty$\\
量 $T_\infty$ (=1000K) は、外気圏温度、すなわち、平均自由行程がスケール高さを超える約 500 km より上で、$T(Z)$ を表す指数関数が漸近線となる値を表す。この標準用に採用された $T_\infty$ の値は、平均的な太陽状態を表すと想定されている。\\
\\$u_i$\\
量 $u_i$ は、$q_i$ adove とともに説明されている6つの定数のセットの3番目のセットを表している。\\
\\$U_i$\\
量 $U_i$ は、$q_i$ adove とともに説明されている6つの定数のセットの4番目のセットを表している。\\
\\$w_i$\\
量 $w_i$ は、上記の $q$ とともに説明されている6つの定数のセットの5番目のセットを表している。\\
\\$W_i$\\
量 $W_i$ は、$q$ とともに説明されている6つの定数のセットの6番目のセットを表している。\\
\\$Z_b$\\
量 $Z_b$ は、$b$ が 7 から 12 に等しい $Z$ の6つの値のセットを表している。値 $Z_7$, $Z_8$, $Z_9$、および $Z_{10}$ は、この標準の採用された温度-高さ関数の連続するセグメントによって特徴付けられる連続する層の基部に対応している。5番目の値 $Z_{11}$ は、原子状水素計算の基準高さであり、6番目の値 $Z_{12}$ は、標準の表形式の値が与えられる領域の上部を表している。これらの $Z_b$ の6つの値は、これらの値の最初の4つに関連付けられた温度-高さ関数のタイプとともに、線形の温度-高さ関数を持つ2つのセグメントの関連する値 $L_{K,b}$ とともに、表5にリストされている。\\
\\$\alpha_i$\\
量 $\alpha_i$ は、表6にリストされている6つの採用された種に依存する熱拡散係数のセットを表している。量 $\sigma^* = (10^{-3}\, \mathrm{K/cm})$ は、$T_{0} \sim 100$ K の古典的な「最小フラックス」フラックスであり、臨界レベルでのマクスウェル速度分布からの逸脱を考慮に入れる補正(Brinkman 1971)と、プラズマ圏での $R^+$ および $O^-$ との電子交換の影響(Tinsley 1970)を考慮に入れるために選択された項である。\\
\\$\phi$\\
$T _\infty$ = 1000 K の垂直フラックスの量 $\phi$ (= $7.2\times 10^{11}\text{m}^{-2} \cdot \text{s}^{-1}$)。臨界レベルでの偏差を考慮に入れる補正 (Brink-man 1971)、およびプラズマ圏での $\text{H}^+$ および $\text{O}^+$ との電荷交換の影響 (Tinsley 1973)。

\subsubsection{平衡条件}
空気は乾燥していると仮定され、86 km を十分に下回る高さでは、大気は均一に混合されていると仮定され、一定の平均分子量 $M$ につながる相対分子組成を持つ。
空気が完全な気体として扱われる場合、大気中の任意の点での全圧力 $P$、温度 $T$、および全密度 $\rho$ は、状態方程式、すなわち完全な気体法則によって関連付けられ、その1つの形式は

\begin{equation}
  P = \frac{\rho \cdot R^* \cdot pT}{M} \tag{1}
\end{equation}

ここで、$R^*$ は普遍気体定数である。
状態方程式の代替形式は、全数密度 $N$ とアボガドロ定数 $N_A$ を用いて

\begin{equation}
  P = \frac{N \cdot R^* \cdot T}{N_A} \tag{2}
\end{equation}

この形式は、$P_i$ の合計を表しており、これは個々のガス種の分圧であり、$P_i$ は次の式で $i$ 番目のガス種の数密度に関連付けられている:

\begin{equation}
    P_i = n_i \cdot k \cdot T \tag{3}
\end{equation}

ここで、$k$ はボルツマン定数である。

完全に混合された高さ領域内では、大気は静水圧平衡状態にあると仮定され、水平方向に成層されているため、圧力の微分 $dP$ は、幾何学的高さの微分 $dz$ に、次の関係によって関連付けられている

\begin{equation}
    dP = -g \cdot \rho \cdot dZ \tag{4}
\end{equation}

ここで、$g$ は高さに依存する重力加速度である。
式 (1) と (4) の関係は、別の形式の静水圧方程式をもたらし、これは低高度圧力計算の基礎となる可能性がある:

\begin{equation}
d \ln P = \frac{dP}{P} = \frac{-g \cdot M}{R^* \cdot T} \cdot dZ \tag{5}
\end{equation}

86 km を超えると、個々のガス種の拡散と垂直輸送が拡散分離を含む動的に調整されたモデルの必要性につながるため、大気の静水圧平衡は徐々に崩れる。
これらの条件下では、個々のガス種の分子のフラックスの垂直成分の観点から、大気数密度の高さの変化を表すのが便利である(Colgrove et al. 1966)。$i$ 番目のガス種の観点から、この式は

\begin{equation}
  n_i \cdot v_i  + D_i \cdot \left ( \frac{dn_i}{dz} + \frac{n_i}{T} \cdot \frac{dT}{dz} + \frac{n_i}{M} \cdot \frac{dM}{dz}  + \frac{g \cdot n_i \cdot m_i}{R^* \cdot T}  \right ) + K \cdot \left ( \frac{dn_i}{dz} + \frac{n_i}{T} \cdot \frac{dT}{dz}  + \frac{g \cdot n_i \cdot m_i}{R^* \cdot T} \right ) = 0 \tag{6}
\end{equation}\\
ここで\\
$v_i$ = $i$ 番目の種の垂直輸送速度,\\
$D_i$ = $N_2$ を通して拡散する $i$ 番目の種の高さに依存する分子拡散係数,\\
$a_i$ = $i$ 番目の種の熱拡散係数.\\
$M_i$ = $i$ 番目の種の分子量,\\
$M$ = $i$ 番目の種が拡散している気体の分子量、および\\
$K$ = 高さ依存の渦拡散係数。\\
\\関数 $K$ は、3つの高さ領域のそれぞれで異なる方法で定義される:\\
\begin{enumerate}
    \item $86 \leq Z < 95$ km の場合、
    \[
    K = K_7 = 1.2 \times 10^2 \, \text{m}^2/\text{s} \tag{7a}
    \]

    \item $95 \leq Z < 115$ km の場合、
    \[
    K = K_7 \cdot \exp\left[1 - \frac{400}{400 - (Z - 95)^2}\right] \tag{7b}
    \]

    \item $115 \leq Z < 1000$ km の場合、
    \[
    K = K_{10} = 0.0 \, \text{m}^2/\text{s} \tag{7c}
    \]
\end{enumerate}

関数 $D_i$ は次のように定義される
\[
D_i = \frac{a_i}{\Sigma n_i} \cdot \left(\frac{T}{273.15}\right)^{b_i} \tag{8}
\]

ここで、$a_i$ と $b_i$ は表6で定義されている種に依存する定数であり、$T$ と $\Sigma n_i$ は両方とも高度に依存する量であり、以下で詳細に指定されている。これらの高度に依存する量と定義された定数 $a_i$ と $b_i$ から決定された $D_i$ の値は、4つの種である O、O$_2$、Ar、および He のそれぞれについて、高度の関数として図1にプロットされている。原子状水素の $D_i$ の値も、150 km のすぐ下の高さについて、図1に示されている。この同じ図には、$K$ の高度の関数としてのグラフが含まれている。90 km を十分に下回る高さでは、$D_i$ の値は $K$ と比較して無視できるほど小さく、115 km を超えると、その逆が真であることが明らかである。さらに、さまざまな種のフラックス速度 $v_i$ は、90 km を十分に下回る高度では無視できるほど小さくなることが知られている。

$V_i$、$D_i$、および $K$ の相対的な大きさに関する情報により、いくつかの体制のそれぞれで式 (6) の適用を検討することができる。

これらの体制の1つは、90 km を十分に下回る高さの場合であり、$v_i$ と $D_i$ の両方が $K$ と比較して非常に小さい。これらの条件下では、式 (6) は静水圧方程式の次の形式に簡略化される:

\[
\frac{d n_i}{n_i} + \frac{dT}{T} = -\frac{g \cdot M}{R^* \cdot T} \cdot dZ. \tag{9}
\]

この方程式の左辺は、式 (3) を通して $dP_i / P_i$ に等しいことがわかっているため、式 (9) は式 (5) に相当する単一ガスであることがわかる。したがって、式 (6) は86 km を超える個々のガスの想定される平衡状態を記述するように設計されているが、式 (6) はその高度を下回る状態も記述していることは明らかであり、全体の圧力を構成する各ガスの分圧は、混合物の平均分子量、ならびに温度と重力加速度に応じて変化する。それにもかかわらず、全圧を表す式 (5) は、$T$、$M$、および $g$ の高度変化を考慮に入れるために適切な関数が導入されたときに、幾何学的高さに対する全圧を計算するための方程式の開発における便利なステップを表している。

標準大気計算では、独立変数 $Z$ を地勢ポテンシャル高度 $H$ に変換することにより、重力加速度の可変部分を式 (5) から効果的に排除し、それにより式 (5) の積分と圧力計算の結果の式を簡略化することが慣例となっている。幾何学的高度と地勢ポテンシャル高度の関係は、重力の概念に依存する。

\subsubsection{重力と地勢ポテンシャル高度}

地球に固定された参照フレームから、通常の方法で見ると、大気は重力の影響を受ける。重力は、次の2つの力の合力(ベクトル和)である。
\begin{enumerate}
    \item ニュートンの万有引力の法則に従った重力引力、および
    \item 地球に固定された回転参照フレームの選択から生じる遠心力。
\end{enumerate}

重力場は保存場であるため、単位質量あたりの重力ポテンシャルエネルギー、つまり地勢ポテンシャル $\Phi$ から導出できると便利である。これは次のように与えられる
\[
\Phi = \Phi_G + \Phi_C \tag{10}
\]
ここで、$\Phi_G$ は重力引力の単位質量あたりのポテンシャルエネルギーであり、$\Phi_C$ は遠心力に関連付けられた単位質量あたりのポテンシャルエネルギーである。単位質量あたりの重力は
\[
 \text{g} = \nabla \Phi \tag{11}
\]
ここで、$\nabla \Phi$ は地勢ポテンシャルの勾配(上昇)である。

重力による加速度は $g$ で表され、$g$ の大きさとして定義される。つまり

\begin{equation}
 g = |\text{g}| = |\nabla \phi|. \tag{12}
\end{equation}

表面 $\phi_0$ 上の任意の点から、最初の表面に近い表面 $\phi_1+d\phi$ 上の点まで外部法線に沿って移動すると、$\phi_1 = \phi_0 + d\phi$ となるため、単位質量を最初の表面から2番目の表面にシフトすることによって行われる増分作業は

\begin{equation}
 d\phi = g \cdot dZ. \tag{13}
\end{equation}

したがって

\begin{equation}
 \phi = \int_{0}^{Z} g \cdot dZ . \tag{14}
\end{equation}

地勢ポテンシャルの測定単位は、標準地勢ポテンシャルメートル (m') であり、これは重力加速度が均一に 9.80665 m/s$^2$ である領域を介して、単位質量を 1 ジオメーターメートル持ち上げることで行われる作業を表す。

平均海面(ゼロポテンシャルと仮定)に関する任意の点の地勢ポテンシャル $\phi$ は、地勢ポテンシャルメートルで表され、地勢ポテンシャル高度として定義される。
したがって、地勢ポテンシャル高度 $H$ は次のように与えられる

\begin{equation}
 H = g_0^{-1} \int g \cdot dZ. \tag{15}
\end{equation}

単位地勢ポテンシャル $g_0^{-1}$ が 9.80665 m'/s$^2$ に等しく設定されている場合、地勢ポテンシャルメートル (m') で表される。

地勢ポテンシャル高度が式 (15) で定義されている場合、$H$ の微分 ($dH$) は次のように表すことができる

\begin{equation}
 g\cdot dH = g\cdot dZ. \tag{16}
\end{equation}

この式は、その積分プロセスの前に変数の数を減らすために式 (5) で使用され、地勢ポテンシャル高さの関数として圧力を計算するための式につながる。

万有引力の逆二乗の法則は、$g$ を高度の関数として表す式を提供する。これは、ほとんどのモデル大気計算に十分な精度である

\begin{equation}
 g = g_0 \left( \frac{r_0}{r_0 + Z} \right)^2. \tag{17}
\end{equation}

ここで、$r_0$ は、ランベルトの方程式 (List 1963) で与えられる特定の緯度での地球の有効半径である。
このような $r_0$ の値は、特定の緯度での遠心加速度を考慮に入れている。
この標準では、$r_0$ の値は $6.356766 \times 10^6$ m とされ、重力加速度の海面値として採用された値 $g_0 = 9.80665 \, \text{m/s}^2$ と一致する。
幾何学的高度の関数としての $g$ の変動を図2に示す。

式 (15) の積分は、$g$ に式 (17) を代入した後、次のようになる
\begin{equation}
 H = g_0 \cdot \frac{r_0}{g_0 \cdot r_0 + Z} \cdot \frac{r_0 \cdot Z}{r_0 + Z}. \tag{18}
\end{equation}
または
\begin{equation}
 Z = \frac{r_0 \cdot H}{r_0 - H}. \tag{19}
\end{equation}

ここで、$\gamma = g/g_0 = 1$ m'/m。

Z のさまざまな値について式 (18) から得られた地勢ポテンシャル高度と、1962 年の米国標準大気の開発で使用されたより複雑な関係から計算された地勢ポテンシャル高度の差は小さい。たとえば、式 (18) から計算された H の値は、1962 年の標準で使用された関係から得られた値よりも、90、150、および 700 km でそれぞれ約 0.2、3.4、および 33.5 m 大きい。

式 (18) の Z から H への変換により、表面から 86 km までの T の変動だけでなく、M の変動を利用する必要がなくなる。また、H の項を定義した。したがって、表面から 86 km までの M の海面値とこの量の高さ依存性の範囲を決定するのが便利である。したがって、この低高度体制では、2つの変数 $T$ と $M$ は定数 $\bar{M}$ と組み合わされ、単一の変数 $T_M$ になり、これは $H$ の関数として定義される。

\subsubsection{平均分子量}
気体の混合物の平均分子量 $\bar{M}$ は、定義により

\begin{equation}
 \bar{M} = \frac{\sum (n_i \cdot M_i)}{\sum n_i} \tag{20}
\end{equation}

ここで、$n_i$ と $M_i$ は、それぞれ $i$ 番目のガス種の数密度と定義された分子量である。表面から約 86 km の高度までの大気の部分では、混合が支配的であり、拡散と光化学プロセスが $\bar{M}$ に及ぼす影響は無視できる。この領域では、種の分数組成 $F_i$ は定義された値 $F_i$ で一定のままであると仮定され、$\bar{M}$ はその海面値 $\bar{M}_0$ で一定のままである。これらの条件では、$n_i$ は $F_i$ と全数密度 $N$ の積に等しいため、式 (20) は次のように書き換えることができる

\begin{equation}
 \bar{M} = \frac{\sum [F_i \cdot N \cdot M_i]}{\sum [F_i \cdot N]} = \frac{\sum [F_i \cdot M_i] \cdot N}{\sum [F_i] \cdot N} = \frac{\sum [F_i \cdot M_i]}{\sum F_i} \tag{21}
\end{equation}

この方程式の右側の要素は、先行する方程式の分子と分母の両方の各項から $N$ を因数分解するプロセスから生じるため、N の高度依存性にもかかわらず、$\bar{M}$ は分析的に完全な混合の高度領域全体で $\bar{M}_0$ に等しいことがわかる。

定義された $F_i$ と $M_i$ の値(表3から)が式 (21) に導入されると、$\bar{M}_0$ は 28.9644 kg/kmol であることがわかる。ただし、86 km(84.852 km)では、原子状酸素の数密度(8.6 x 10$^{13}$ /m$^3$)の定義された値は、付録 A で $\bar{M} = 28.9522$ kg/kmol の値につながることがわかり、$\bar{M}_0$ よりも約 0.04% 少ない。

海面でのこの初期値から 86 km での値へのスムーズな移行を生み出すために、70.000 km から 84.852 km までの高度について、0.5 km 間隔で、表7に示すように、比 $\bar{M} / \bar{M}_0$ の観点から任意に定義されている。これらの比の値は、80 km で $\bar{M} = \bar{M}_0 = 28.9644$ kg/kmol、86 km で $\bar{M} = 28.9522$ の境界条件を満たし、この高さ間隔 80 km から 86 km の範囲で $\bar{M}$ の滑らかに減少する一次差分の条件を満たすように、最初は運動温度の 80 km から 86 km の間隔で選択された値から内挿されている。

これらの任意に割り当てられた値 $\bar{M} / \bar{M}_0$ は、この標準の多くのパラメータを修正するために使用される場合がある。表の記述が、この高さ領域内で5番目、場合によっては4番目の有効数字でモデルに正しく適合するようにする場合である。

この事後修正が必要なのは、これらの $\bar{M}$ の値が86 km未満のこの標準の表を計算するために使用されたプログラムに含まれていなかったためであり、したがって、特性の一部は、85.5 km と 86 km の間で最大 0.04% の不連続性を示す場合があるためである。この状況は、特に4つの特性で存在する。分子量、すなわち運動温度、全数密度、平均自由行程、および衝突頻度に加えて。

これらの5つのパラメータについては、80 km と 86 km の間の表の不一致は、表形式の $T$、$M$、および $L$ の値を、表形式の $\bar{M}$ の対応する値で単純に乗算または除算し、$\bar{M}_0$ の対応する値で除算するだけで、簡単になくすことができる。

動粘度、運動粘度、および熱伝導率の他の3つの特性は、86 km 未満の高さでのみ表形式で示されており、86 km のすぐ下の高さについても同様の不一致がある。

ただし、これらの値は、それぞれの定義関数の経験的な性質のために、それほど簡単に修正できない。むしろ、80 km から 86 km の間の幾何学的高度で精度の高い正しい値が必要な場合は、適切に修正された $T$ の値のセットの観点から、これらの量を再計算する必要がある。

\subsubsection{分子スケール温度対地勢ポテンシャル高度 (1.0 ~ 44.8525 KM)}

点での分子スケール温度 $T_M$ (Minzner et al. 1958) は、運動温度 $T$ と $\bar{M}_0$ から $\bar{M}$ への比の積として定義される。ここで、$\bar{M}$ はその点での空気の平均分子量であり、$\bar{M}_0$ は上記の $\bar{M}$ の平均海面値である 28.9644 である。分析的に、
\begin{equation}
  T_M = T \cdot \frac{\bar{M}_0}{\bar{M}}. \tag{22}
\end{equation}

$T$ がケルビンスケールで表される場合、$T_M$ もケルビンスケールで表される。

パラメータ $T_M$ の原則的な値は、$M$ の可変部分を変数 $T$ と組み合わせて、新しい単一の変数にすることであり、変数の $g$ の可変部分と $Z$ を組み合わせて新しい変数 $H$ を形成するのと多少似ている。
これらの変換の両方が (5) に導入され、$T_M$ が線形として表現される場合、正確な積分を持つ。これらの条件下では、$P$ 対 $H$ の計算は、数値積分を必要としない簡単なプロセスになる。伝統的に、標準大気は、高さに対する圧力を計算する際に数値積分を不要にするために、温度を高さの線形関数として定義してきた。この標準は、最大 86km までの高さの伝統に従い、$T_M$ 対 $H$ の関数は、一連の7つの連続する線形方程式として表される。これらの線形方程式の一般的な形式は
\begin{equation}
  T_M = T_{M,b} + L_{M,b}\cdot (H - H_b)
  \tag{23}
\end{equation}

ここで、添え字 $b$ の値は、7つの連続するレイヤーのそれぞれに応じて 0 から 6 の範囲である。
最初のレイヤー ( $b$ = 0) の $T_{M,b}$ の値は 288.15$K$ であり、$T_0$、すなわち $T$ の海面値と同一である。なぜなら、このレベルでは $M$ = $M_0$ であるからである。$T_{M,b}$ のこの値が定義され、表 4 で定義されている $H_0$ の6つの値のセットと、対応する $L_{M,b}$ の6つの値が定義されている場合、$H$ の関数 $T_M$ は、表面から $84.8520 \, \text{km}^{\prime}$ (86 km) まで完全に定義される。この関数のグラフは、図3で1962年標準の同様の関数と比較される。表面から $51-\text{km}^\prime$ 高度まで、このプロファイルは1962年標準のものと同一である。51 から $84.8520 \text{km}^\prime$ までのプロファイルは、タスクグループIによって選択され、このプロファイルに基づいた大気の熱力学的特性の短縮された表がKantor and Cole (1973) によって公開された。

\subsubsection{運動温度対幾何高度 0.0 ~ 10000km}
表面から 86 km の高度の間では、運動温度は $T_M$ の定義された値に基づいている。$M$ が $M_0$ で一定であるこの領域の最低80キロメートルでは、$T$ は (22) に従って $T_M$ に等しい。ただし、80 km と 86 km の間では、比 $M/M_0$ は表 8 に示すように、1.000000 から 0.9995788 まで減少すると想定されており、対応する $T$ の値は $T_M$ の値から減少する。したがって、$Z_7 = 86$km での式 (22) の形式は、$T_7$ が 186.8673 K の値を持つこと、すなわち、tgat 高さでの $T_M$ の値よりも 0.0787 K 小さいことを示している。

86km より高い高度では、$T_M$ の値は定義されなくなり、地勢ポテンシャルはもはや主要なものではない。4つの連続する関数の観点から定義されており、これらの関数のそれぞれは、$Z$ に関する $T$ の1次導関数が、86 から 1000km までの高度領域全体で連続的になるように指定されている。これらの4つの関数は、表 5 にリストされている最初の4つの基部高さ $Z_b$ で連続的に始まり、次の条件を表すように設計されている:\\
\indent A. 86 から 91 km までの等温層。\\
\indent B. 91 から 110km まで、$T(Z)$ が楕円の形式を持つ層。\\
\indent C. 110 から 120km までの一定の正勾配層。および\\
\indent D. $Z$ が 120 から 1000km まで増加するにつれて、$T$ が指数関数的に漸近線に向かって増加する層。
86から91km\\

$Z_7$ = 86km から $Z_8$ = 91km までの層の場合、温度-高度関数は、幾何高度に関して等温的に線形であると定義されるため、$Z$ に関する $T$ の勾配はゼロである(表5を参照)。したがって、線形関数の標準形式は
\begin{equation}
  T = T_b + L_{M,b}\cdot (Z - Z_b)
  \tag{24}
\end{equation}
次のように縮退する
\begin{equation}
  T= T_7 = 186.8673 K
  \tag{25}
\end{equation}
定義により
\begin{equation}
  \frac{dT}{dZ} = 0.0 \text{K/km}
  \tag{26}
\end{equation}

$T_7$ の値は、式 (22) のあるバージョンから導出され、そこでは $T_M$ が $T_{M7}$ に置き換えられ、1.2.5 adove で決定された値であり、$M/M_0$ が $M_7/M_0$ に置き換えられ、1.3.3 で説明されている $M_0$ と $M_7$ の値に従って、0.8885788 の値を持つ。$T$ は層全体、$Z_7$ から $Z_8$ まで一定であると定義されているため、$Z_8$ での温度は $T_8 = T_7 = 186.8673$K であり、$Z_8$ での勾配 $dT/dZ$ は $L_{K,8} = 0.0\text{K/km}$ であり、$L_{K,7}$ と同じである。\\\\
91 ~ 110 km

層 $Z_8$ = 91 km から $Z_9$ = 110km の場合、温度-高度関数は、次のように表される楕円の一部のセグメントであると定義される
\begin{equation}
  T = T_c + A \cdot \left[ 1 - \left( \frac{Z - Z_8}{a} \right)^2 \right]^{1/2}
  \tag{27}
\end{equation}
ここで
$T_c = 263.1905$ K, $A = -76.3232$ K, $a = -19.9429$ km, であり、$Z$ は 91 km から 110 km までの値に制限されている。

式 (27) は、楕円の基本方程式から付録 B で導出され、上記の $T_s$ と $L_{K,s}$ の値、ならびに $Z_9 = 110~\text{km}$ に対する定義された値 $T_9 = 240.0~\text{K}$ と $L_{K,9} = 12.0~\text{K/km}$ を満たす。

式 (27) に関連する $dT/dZ$ の式は
\[
\frac{dT}{dZ} = -\frac{A}{a} \left[ \left(\frac{Z - Z_s}{a} \right) \left(1 - \left(\frac{Z - Z_s}{a} \right)^2 \right)^{-1/2} \right]. \tag{28}
\]

層 $Z_9 = 110~\text{km}$ から $Z_{10} = 120~\text{km}$ の場合、$T(Z)$ は (24) の形式を持ち、添え字 9 はそのようなものであり、$T_9$ と $L_{K,9}$ はそれぞれ定義された量 $T_9$ と $L_{K,9}$ であり、$Z$ は 110 から 120 km の範囲に制限されている。したがって、
\[
T = T_9 + L_{K,9} \, (Z - Z_9) \tag{29}
\]
そして
\[
\frac{dT}{dZ} = L_{K,9} = 12.0~\text{K/km}. \tag{30}
\]

$dT/dZ$ は層全体で一定であるため、$Z_{10}$ での $dT/dZ$ の値である $L_{K,10}$ は、$L_{K,9}$、つまり $12.0~\text{K/km}$ と同一であり、$Z_{10}$ での $T$ の値は式 (29) から $360.0~\text{K}$ であることがわかる。

120 ~ 1000 km

層 $Z_{10} = 120~\text{km}$ から $Z_{12} = 1000~\text{km}$ の場合、$T(Z)$ は指数関数形式 (Walker 1965) を持つと定義される
\[
T = T_{12} - (T_{12} - T_{10}) \cdot \exp(-\lambda \, \xi) \tag{31}
\]
そのように
\[
\frac{dT}{dZ} = \lambda \cdot (T_{12} - T_{10}) \cdot \left(\frac{r_0 + Z_{10}}{r_0 + Z}\right)^2 \cdot \exp(-\lambda \, \xi) \tag{32}
\]
ここで
\[
\lambda = \frac{L_{K,12}}{(T_{12} - T_{10})} = 0.01875,
\]
そして
\[
\xi = \xi(Z) = \left(\frac{Z - Z_{10}}{r_0 + Z_{10}}\right) \cdot (r_0 + Z).
\]

上記式では、$T_{12}$ は定義された値 $1000~\text{K}$ に等しい。高度 $0.0$ から $100~\text{km}$ までの $T$ 対 $Z$ のグラフを図4に示す。このプロファイルの上のプロファイルは、タスクグループ III によって選択され、衛星ドラッグデータ (Jacchia 1971) と一致するようにされた whike、中央部分、特に 86 km から 200 km の間、および 450 km までのオーバーラップは、観測された温度と組成データの衛星観測 (Hedin et al. 1972) と一致するように、タスクグループ II (Minzner et al. 1974) によって選択された。

\subsection{計算式}

この標準の表は、2つの高さ領域、$0$ から $84.852 \, \text{km}$ (86 km)、および $86$ から $1000 \, \text{km}$ で計算されている。各領域の計算は、互換性のある異なる初期条件のセットに基づいているためである。これらの2つの異なる初期条件のセットは、2つの異なる計算手順につながる。その結果、一連の大気パラメータに従って提示される計算式に関する次の説明は、各高度領域で実際に実行される計算の順序に必ずしも従わない。86 km 未満の高度の大気のさまざまな特性を計算するために使用される式は、特定の例外を除いて、1962 年の標準で使用された式と同等であり、$T_m$ を含むさまざまな式は、ARDC モデル大気、1956 (Minzner and Ripley 1956) で使用された式から来ている。

\subsubsection{圧力}

この標準のさまざまな高さ体制で圧力を計算するために、3つの異なる式が使用されている。これらの式の1つは 86 km を超える高さに適用され、他の2つは表面から 86 km までの高さ体制に適用され、計算の引数は地勢ポテンシャルである。その結果、地勢ポテンシャル高度の関数として圧力を計算するための式は、式 (16) から $g \cdot dZ$ をその同等の $g'_0 \cdot dH$ に置き換えて、また式 (22) に従って比 $M/T$ をその同等の $M_b/T_{m,b}$ に置き換えた後、式 (5) の積分から生じる。2つの形式はこの積分から生じ、1つは特定の層の $L_{m,b}$ がゼロに等しくない場合、もう1つは値 $L_{m,b}$ がゼロである場合である。これらの2つの式の最初の式は

\[
P = P_b \cdot \frac{T_{m,b}}{T_{m,b} + L_{m,b} \cdot (H - H_b)} \exp \left[ \frac{g'_0 \cdot M_b}{R^* \cdot L_{m,b}} \right] \tag{33a}
\]

後者は

\[
P = P_b \cdot \exp \left[ \frac{-g'_0 \cdot M_b \cdot (H - H_b)}{R^* \cdot T_{m,b}} \right] \tag{33b}
\]

これらの式では、$g'_0$、$M_b$、および $R^*$ はそれぞれ定義された単一値定数であり、$L_{m,b}$ と $H_b$ はそれぞれ表 4 に示されている $b$ の値に従って定義された多値定数である。量 $T_{m,b}$ は、$L_{m,b}$ と $H_b$ に依存する多値定数である。$b = 0$ の場合の $P_b$ の基準レベル値は、定義された海面値である $P_0 = 101325.0 \, \text{N/m}^2$ (1013.250 mb) である。$b = 1$ から $b = 6$ までの $P_b$ の値は、$H = H_{b+1}$ の場合の式 (33a) と (33b) のペアの適切なメンバーの適用から得られる。

これらの2つの方程式は、海面から $H_t$ までの任意の望ましい地勢ポテンシャル高度の圧力を生成する。ここで、$H_t$ は幾何学的高度 $Z_t = 86 \, \text{km}$ に対応する地勢ポテンシャル高度である。$0$ から $-5 \, \text{km}$ までの $H$ の圧力も、添え字 $b$ がゼロの場合、式 (33a) から計算できる。

$Z$ が $86 \, \text{km}$ 以上に等しい場合、圧力の値は幾何学的高度 $Z$ の関数として計算され、式 (3) で表される個々の種の分圧の合計に全圧 $P$ が等しい式で、$T_m$ ではなく、運動温度 $T$ の高度プロファイルに関与する。したがって、$Z = 86$ から $1000 \, \text{km}$ の場合、

\[
P = \Sigma P_i = \Sigma n_i \cdot k \cdot T = \frac{\Sigma n_i \cdot R^* \cdot T}{N_A} \tag{33c}
\]

この式では

\begin{itemize}
    \item $k$ = ボルツマン定数、表 2a で定義、
    \item $T$ = $T(Z)$ は、連続する層の式 (25)、(27)、(29)、および (31) で定義、および
    \item $\Sigma n_i$ = 上記で説明したように、86 km を超える高度 $Z$ で大気を構成する個々のガス種の数密度の合計。
\end{itemize}

個々の種族の数密度である $n_i$ も、個々の数密度の合計である $\Sigma n_i$ も、直接的には知られていない。その結果、86 km を超える圧力は、重要なガス種のそれぞれについて $n_i$ を最初に決定しないと計算できない。
\end{document}